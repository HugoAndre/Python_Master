\section*{TP : Introduction à la gestion de versions avec Git}
\textbf{Objectifs} : à l'issue de ce TP, vous devrez savoir
\begin{itemize}
\item versionner votre travail en local
\item annuler des modifications
\item manipuler les branches pour isoler des développements
\item interagir avec un dépôt distant
\item résoudre des conflits simples
\end{itemize}


\textbf{Liens }: il est fortement recommandé de se documenter tout au long du TP

\begin{itemize}
\item \url{https://git-scm.com/doc} et \url{https://git-scm.com/book/fr/v2}
\item man git ou git help nom\_commande
\end{itemize}

\textbf{Note} : Ce TP couvre les mêmes points que gittutorial (man gittutorial). Vous pouvez le lire en parallèle pour
avoir des explications complémentaires ou le faire ultérieurement. Ceux qui finissent le TP en avance
peuvent poursuivre avec gittutorial-2.

\section{Exercice 1 : Dépôt local}
\section{Configuration initiale}
Nous allons tout d'abord renseigner quelques informations qui seront inscrites dans vos commits.
\mint{python}| git config --global --add user.name "Prénom Nom"|
\mint{python}| git config --global --add user.email "prenom.nom@telecomnancy.eu"|
\mint{python}| git config --global merge.conflictstyle diff3|
Vérifier la prise en compte de la configuration : 
\mint{python}|cat ~/.gitconfig|
Récupérer un .gitignore proposé par github, par exmple celui pour TeX :
\url{https://github.com/github/gitignore}\\
Un .gitignore peut être défini soit localement pour chaque dépôt, soit globalement :
\mint{python}| git config --global core.excludesfile ~/.gitignore_global|

\subsection{Versionner un répertoire local}
Créer un répertoire local : \mint{python}|mkdir tp_git|, et faites en votre répertoire courant : \mint{python}|cd tp_git|
Versionner ce répertoire : \mint{python}|git init|, vérifier que le répertoire contenant la base de données de git
« .git » a bien été créé à la racine : \mint{python}|ls -la .git|
Ajouter deux fichiers texte dans le répertoire : recipe\_appbread.txt, recipe\_soup.txt .\\
Indexer ces deux fichiers :\mint{python}| git add nom_fichier|
Vérifier la prise en compte en observant l'état de l'index (zone de staging) : \mint{python}|git status|
Enregistrer cette version dans le dépôt : \mint{python}|git commit|
git ouvre alors un éditeur de texte et vous invite à renseigner le message du commit, entrer un
message explicite « Ajout des recettes de soupe à l'oignon et d'apple bread ». L'option git
commit -m "message" permet de tout faire en une commande.\\
Vérifier la prise en compte du commit en observant le log : git log , observer l'identifiant SHA1.\\
\subsection{Observer les modifications en cours}
Faire une première modification dans un fichier et sauvegarder.\\
Observer les modifications perçues par git grâce à git status, git diff et git diff --cached\\
Indexer le fichier précédemment modifié : 
\mint{python}|git add fichiers|
Observer à nouveau avec les mêmes commandes afin de percevoir ce que fait chacune.
Commiter cette modification : 
\mint{python}|git commit -m "message"\ et observer une dernière fois.|
\section{Annuler ou corriger des modifications}
\subsection{ Annuler une modification du working directory}
Modifier et sauvegarder un fichier.\\
 Cette modification est une erreur, rétablir le fichier tel que définit dans un commit (le dernier) :
\mint{python}|git checkout SHA1\_ancien\_commit nom\_fichier|
\subsection{Annuler une modification indexée}
Modifier et sauvegarder un fichier. Indexer la modification.\\
Modifier une nouvelle fois le même fichier. Malheureusement, cette nouvelle modification est
une erreur.\\
Rétablir le fichier dans l'état où il a été précédemment indexé : \mint{python}|git checkout nom_fichier|
Finalement, vous décidez de ne pas enregistrer les modification de ce fichier pour le prochain
commit, désindexer le : \mint{python}|git reset nom_fichier|
Vérifier l'état de l'index : git status\\
Rétablir l'ensemble du dépôt tel que définit dans le dernier commit : git reset --hard\\
\subsection{Annuler une modification enregistrée}
Modifier et sauvegarder un fichier. Indexer puis enregistrer la modification. Vérifier la prise en
compte de l'enregistrement avec git log.\\
 Encore une erreur ! Annuler complètement le dernier commit : \mint{python}|git revert SHA1_dernier_commit|
 Modifier et sauvegarder un fichier. Indexer puis enregistrer la modification.\\
  Vous avez oublié de modifier le second fichier. Le modifier et indexer le changement.\\
   On souhaite maintenant corriger le dernier commit pour y enregistrer les modifications du
second ficher : git commit --amend (corriger si besoin le message du commit)\\
 Vérifier que le dernier commit inclut bien désormais la modification du second fichier :
 \mint{python}|git show HEAD|
\section{Manipuler les branches}
\subsection{Créer une nouvelle branche}
Créer une nouvelle branche de développement : \mint{python}|git branch experimental|
Changer de branche pour la nouvelle : git checkout experimental\\
Lister les branches : git branch\\
Vérifier que la HEAD est bien positionnée sur la nouvelle branche : cat .git/HEAD
\subsection{Merger (fusionner) deux branches}
Ajouter une ligne dans un fichier et le sauvegarder. Indexer puis enregistrer la modification.\\
Vérifier que le commit est bien enregistré : git log\\
Revenir sur la branche principale : git checkout master\\
Observer que la modification n'est plus présente dans le fichier précédemment édité ! C'est
l'intérêt des branches : plusieurs versions co-existent en parallèle.\\
Vérifier que le précédent commit est absent de cette branche : git log\\
Pour intégrer les modification de la branche experimental, il faut faire un merge : \mint{python}|git merge experimental|
Vérifier que les modifications ont bien été intégrées
\subsection{Visualiser les branches}
Utiliser l'interface graphique gitk pour visualiser le graphe d'historique : gitk --all\\
Supprimer la branche : git branch -d experimental
\section{Utiliser un dépôt distant}
A partir de cet exercice, il faut travailler en binôme. Pour plus d'info :
\url{https://git-scm.com/book/fr/v2/Les-branches-avec-Git-Les-branches-distantes}
\subsection{Créer un dépôt sur GITHUB}
Aller sur Github , créez vous un compte au nom explicite: Nom\_Prenom\_UJM.\\
Créer un nouveau projet (repository) appelé "tp\_git\_votrenomdefamille"\\
partagez le lien http à votre binome et à Hugo ANDRE.
\subsection{Lier le dépôt local a un dépôt distant}
Il faut lier votre dépôt local à votre dépôt distant sur la forge :
\mint{html}|git remote add origin 
https://git...tp_git_nom.git|
 « origin » est simplement ici le nom par défaut donné en local à ce dépôt distant\\
  En cas d'erreur, l'adresse du dépôt distant peut être changée :
\mint{python}| git remote set-url origin nouvelle_adresse|
Ou le dépôt distant peut être supprimé : \mint{python}|git remote rm nom_dépôt_distant|
Lister les dépôts distants : git remote show\\
Envoyer vos enregistrements sur le dépôt distant : \mint{python}|git push origin master|
 Si le dépôt distant n'est pas vide, il faut préalablement le merger avec le dépôt local :
\mint{python}| git pull origin master|
L'historique des commits doit maintenant apparaître sur l'interface web de la forge : sous-menu
« Repository » dans le menu de gauche, section « Commits »\\
Parcourez l'ensemble des menus sous l’arborescence « Repository »
\subsection{Récupérer un autre dépôt distant}
Positionnez-vous en dehors de votre dépôt local : cd ..\\
 Récupérer le dépôt distant de votre binôme : git clone \url{https://gitlab.telecomnancy.univlorraine.fr/PrénomBinôme.NomBinôme/tp_git_nombinôme.git}\\
 Ajouter une ligne dans un fichier et le sauvegarder. Indexer puis enregistrer la modification.\\
  Il faut maintenant propager les modification enregistrées localement au dépôt disant :
\mint{python}|git push|
Point de synchronisation pour un binôme
\subsection{Récupérer les modifications enregistrées sur le dépôt distant}
Votre binôme a enregistré des modifications sur votre dépôt distant, les récupérer :
\mint{python}|git pull origin master|
Observer l'historique des commits et vérifier que les modifications ont été prises en compte
\section{Gérer les conflits}
\subsection{Générer un conflit}
Créer un conflit en créant une branche « conflit » et en enregistrant deux modifications
différentes d'une même ligne d'un fichier, l'une sur la branche master, l'autre sur celle
nouvellement créée.\\
 Se replacer sur la branche master et fusionner les branches : git merge doit alors vous notifier
que des conflits sont apparus
\subsection{Résoudre un conflit}
Ouvrir le (ou les) fichier(s) incriminé(s), les marqueurs de conflit <<<<< ||||| >>>>> doivent être
présents\\
Résoudre le problème en choisissant pour chaque conflit la modification pertinente ou en
proposant une nouvelle modification. Ne pas oublier de supprimer les symboles marqueurs de
conflit. Indexer le fichier.\\
La résolution du conflit et la validation du merge se fait en enregistrant les modifications. : git
commit
\subsection{Appliquer une stratégie de résolution de conflits}
Supposons que le conflit survienne entre la branche master et une branche liée à un
développement mineur optionnel. Si l'on ne veut résoudre les conflits manuellement, il est
possible d'appliquer une stratégie privilégiant la branche master dans tous les cas.\\
Reprendre la question 1 de cet exercice , en en appliquant une stratégie de résolution :
\mint{python}|git merge -s ours nom_branche|
\section{Autres commandes utiles}
\subsection{Placer un Tag}
Les tags servent à marquer et à sauvegarder une version précise du projet. Placer un tag sur la
version courante du projet :
\mint{python}|git tag -a v1.0|
La commande git tag (sans argument) liste les tags présents\\
L'envoi de tags vers un dépôt distant doit se faire explicitement :
\mint{python}|git push origin v1.0|
 ou \mint{python}|git push origin –tags|
\subsection{Identifier le fautif}
Git permet de visualiser facilement qui a modifié quoi dans un fichier, par exemple de manière à
identifier le développeur ayant introduit un bug : \mint{python}|git blame nom_fichier|
Exécuter cette commande sur un fichier dont certaines lignes ont été modifiées par votre binôme
\subsection{Réécrire l'histoire}
Il peut être intéressant de réduire et transplanter une branche pour simplifier le graphe des
commits. Créer une nouvelle branche et faire au moins deux commits dessus.\\
Transplanter la nouvelle branche sur la branche principale :
\mint{python}|git rebase --interactive master|
sélectionner le premier commit (pick) et fusionner les autres (squash)\\
 Observer le graphe des commits puis propager les modifications sur le dépôt distant\\
 Enfin, observer l'historique des manipulations du graphe :
\mint{python}| git reflog|